\documentclass[a4paper]{report}

\usepackage[utf8]{inputenc}
\usepackage[ngerman]{babel}

\usepackage{graphicx}
\usepackage{color}



\title{Abschlussbericht des Systemtechnikprojektes im Studiengang Systems Engineering}
\author{
  Erik Rother\\
  \and
  Lisa Schade\\
  \and
  Kira Löper\\
}
\date{28.09.2018}

\begin{document}

\maketitle

\tableofcontents

\chapter{Einleitung}
In der Mikrostrukturierung optischer Bauteile werden Diamantwerkzeuge mit sehr kleinen Schneidenradien eingesetzt. Die Detektion des Werkzeugkontaktes kann nur durch eine Kamera realisiert werden. Die genau Positionierung der Kamera und die Veränderung der Position sind sehr zeitaufwändig. Während unseres Projektes haben wir uns mit diesem Problem beschäftigt und als Lösung eine elektrisch gesteuerte Kameraführung mit automatischer Werkzeugdetektion entwickelt. In diesem Bericht möchten wir den Prozess der Entwicklung und unsere Resultate beschreiben. 

\chapter{Stand der Technik}
Die automatische Werkzeug-Kontakt-Detektion dient zur Unterstützung der Werkzeug-Positionierung bei mehreren Prozessen. Im Folgenden werden zwei Prozesse vorgestellt, die durch die automatische Werkzeug-Kontakt-Detektion zeiteffizienter gestaltet werden können. 

\section{Diamant-Mikroschneiden}
Das Diamant-Mikroschneiden dient zur Herstellung von retroreflektierenden Strukturen. 

\section{Diamant-Drehen}

\chapter{Konzeptentwicklung}

Nachdem wir uns mit der Problematik der Kamerapositionierung zur Werkzeugkontakt-Detektion beschäftigt haben, wurden erste Lösungsansätze erarbeitet. Die grundlegende Fragestellung befasste sich damit, welche Art der Kameraführung am besten für das gegebene Problem geeignet ist. Bei der Beurteilung galt es zu überprüfen, ob die Grundstrukturen die gegebenen Anforderungen erfüllen und dabei gleichzeitig wirtschaftlich sind. In der nachfolgenden Tabelle sind vier Grundstrukturen der Kameraführung mit ihren Vor- und Nachteilen aufgelistet. \\ \\

\begin{tabular}{lll}
 
Grundstruktur der Kameraführung & pro & contra \\
 \hline
 \hline
Raumportal & schnell & groß \\
 & präzise & schwer \\
& größere Lasten möglich &\\
  \hline
Gimbal & leicht & Änderung der Führung notwendig \\
 & schnell & \\
& preiswert & \\
 \hline
Parallelkinematik & sehr präzise & sehr teuer\\
&  enorm flexibel & zweite Führung notwendig \\
  \hline
Tragarm & sehr günstig  & Hebelwirkung vermindert Stabilität\\

\end{tabular}
\\ \\
\\ \\
Das Raumportal bietet zwar viele Vorteile um eine Kamera genau und schnell im Raum auszurichten, ist jedoch für unsere Anforderungen nicht optimal, da es sehr viel Platz einnimmt. Es ist wichtig, dass unsere Kameraführung die Arbeit der Maschine nicht beeinträchtigt, sodass wir uns gegen das Raumportal entschieden haben. \\ \\
 Die Parallelkinematik bietet eine sehr präzise Ausrichtung der Position und kann eine Beweglichkeit in allen sechs Freiheitsgraden ermöglichen zum Beispiel in Form eines Hexapods. Die Bewegungen sind jedoch auch sehr klein, sodass wahrscheinlich eine zweite Führung notwendig wäre, um die Position der Kamera zunächst grob einzustellen. Die Parallelkinematik ist auch sehr teuer, sodass wir diese Lösung nicht als wirtschaftlich ansehen. \\ \\
Im Gegensatz dazu wäre ein Tragarm eine sehr preiswerte Lösung. Jedoch könnte es zu Problemen in der Stabilität kommen, wenn die Hebelwirkung an dem Tragarm zu groß wird. Außerdem sind preiswerte Tragarme manuell einzustellen, was nicht den Anforderungen entspricht. \\ \\
Der Gimbal erschien uns als eine gute Grundlage für das Kamera-Positionierungssystem, da er leicht, schnell und preiswert ist. Der Gimbal enthält drei bürstenlose Motoren mit denen sehr kleine Winkeländerungen möglich sind. Den mechanischen Teil der Führung würden wir auf unsere Anforderungen anpassen. Wir haben uns also für diese Variante der Kameraführung entschieden. \\ \\

TO DO ÜBERARBEITUNG

\chapter{Aufbau des Gesamtsystems}
TO DO BILD \\ \\
Folgende Geräte und Bauteile wurden verwendet
\begin{itemize} 
\item Kamera: Guppy F - 146
\item Objektiv: Macro-CCD Lens 4x
\item Laptop mit FireWire-Schnittstelle
\item Raspberry Pi 2 Model B 
\item Bildschirm, Tastatur, Maus (einschließlich Kabel zum Anschließen an den Raspberry Pi)
\item  Crossover-Kabel
\item  TO DO Motoren, ESC ...
\end{itemize}
Die Position der Kamera ist durch ein Gestell mit drei Motoren variabel, wodurch ihr Winkel so eingestellt werden kann, dass sich Werkzeug und Werkstück optimal im Bildausschnitt befinden. Das Gestell kann sowohl an der Werkzeughalterung, als auch über der Werkstückaufnahme montiert werden. Im Rahmen unseres Projektes konnten wir die Montage TO DO realisieren.
Die Kamera ist über eine FireWire-Schnittstelle mit dem Laptop verbunden. Der Laptop speichert das Bildmaterial und stellt es per Ordnerfreigabe dem Raspberry Pi zur Verfügung. Bildschirm, Maus und Tastatur sind an dem Raspberry Pi angeschlossen, sodass der Benutzer mit Hilfe einer graphischen Oberfläche gewünschte Aktionen auswählen kann. Außerdem verfügt der Raspberry Pi über einen Algorithmus, der zusammenhängende Bereiche mit bestimmten Farbwerten erkennt, wodurch die Kamera automatisch dem Werkzeug folgen kann. (TO DO kann sie das?)

\chapter{Datenübertragung}
Die automatische Positionierung der Kamera setzt voraus, dass die Daten über den aktuellen Bildausschnitt bekannt sind. Die Kamera verfügt über eine FireWire-Schnittstelle. FireWire-Schnittstellen besitzen eine hohe Datendurchsatzrate, wodurch ein besonders schnelles Arbeiten möglich ist. FireWire 400 überträgt Datenströme und ist USB 2.0 daher besonders bei der Übertragung von kontinuierlichen Signalen zum Beispiel im Videobereich überlegen. Jedoch hat sich diese Schnittstelle auf dem Markt nicht durchgesetzt, weshalb viele aktuelle Geräte diese Schnittstelle gar nicht besitzen. USB bietet seit USB 3.0 eine so deutlich überlegene Übertragungsgeschwindigkeit, dass auch im Videobereich größere Datenraten übertragen werden können. \\ \\
Zur Verfügung steht uns ein Laptop der Firma Toschiba mit dem Betriebssystem Windows XP. Dieser bietet einen FireWire-Anschluss, sodass wir die Kamera anschließen können. Das Bildmaterial der Kamera wird mit einer Software von MatLab auf dem Laptop angezeigt und abgespeichert. Da dieser Laptop jedoch sehr langsam arbeitet, wird die weitere Verarbeitung der Bilddaten auf einen Raspberry Pi 2 Model B ausgelagert. \\ \\
Zur Übertragung der Bilddaten nutzen wir das Prinzip der Ordnerfreigabe. Das Betriebssystem Windows XP unterstützt Netzwerkfreigaben von sich aus, während der Raspberry Pi ein zusätzliches Programm für diese Funktion benötigt. Zum Einbinden der Windows-Freigabe nutzen wir das virtuelle Dateisystem \glqq cifs-vfs\grqq. Dazu wird das Paket \glqq cifs-utils\grqq auf dem Raspberry Pi installiert. Um den freigegebenen Ordner in das Dateisystem einzubinden nutzen wir den Befehl \glqq mount\grqq, wobei auch der Typ \glqq cifs\grqq und Angaben über den Ort des freigegebenen Ordners, sowie des Zielordners notwendig sind. Damit man den Ordner nicht bei jedem Neustart des Raspberry Pi's neu einbinden muss, hinterlegen wir alle Informationen für das Einbinden in der Konfigurations-Datei \glqq /etc/fstab\grqq. Nun müssen nur noch mit einem Befehl alle Dateisysteme, die in \glqq /etc/fstab\grqq vermerkt sind, einbinden. Dieser Befehl wird bei öffnen der graphischen Oberfläche unseres Programms automatisch ausgeführt. \\ \\
Für die Verbindung zwischen Laptop und Raspberry Pi gibt es mehrere Möglichkeiten. Wir haben uns mit den zwei Möglichkeiten beschäftigt. \\ \\
 Die erste Möglichkeit ist die Verbindung über das Netzwerk im LFM. Beide Geräte sind durch ein Ethernet-Kabel mit dem Netzwerk verbunden. Ein Vorteil ist, dass die Geräte auch so Zugang zum Internet haben. Der Laptop benötigt den Zugang zum Internet um die notwendigen Lizenzen für MatLab laden zu können. Der freigegebene Ordner kann nun in dem Netzwerk durch Angabe der IP-Adresse des Laptops im Netzwerk und Angabe des Freigabenamens gefunden werden. Dieser Ordner ist nun mit jedem Gerät in dem Netzwerk auffindbar, jedoch ist der Zugriff durch das Passwort des Benutzerkontos von dem Laptop geschützt. \\ \\
Die zweite Möglichkeit ist eine direkte Verbindung durch ein Crossover-Kabel. Diese Verbindung ist sicherer und die Daten können schneller übertragen werden. Jedoch ergibt sich das Problem, dass das Crossover-Kabel den gleichen Anschluss verwendet, wie das Ethernet-Kabel. So kann der Laptop nicht gleichzeitig mit dem Raspberry Pi verbunden sein und Zugang zum Internet haben. Den Internetzugang braucht er allerdings, um die MatLab-Lizenzen zu laden. So müsste zuerst das Ethernetkabel angeschlossen werde, sodass die Lizenzen laden können. Danach müsste man das Kabel entfernen und das Crossoverkabel anschließen, um die Verbindung zum Raspberry Pi herzustellen. Dies wäre nicht besonders anwenderfreundlich. Jedoch konnte diese Möglichkeit optimiert werden, indem durch einen Adapter ein zweiter Netzwerkanschluss an dem Laptop angebracht wurde. Dies ermöglicht Internetzugang mit einem Ethernet-Kabel und mit dem Crossoverkabel diei Verbindung zum Raspberry Pi. Die IP-Adressen werden bei dieser Variante statisch gesetzt. So hat der Laptop die IP-Adresse \glqq 192.168.1.1\grqq und der Raspberry Pi die IP-Adresse  \glqq 192.168.1.2\grqq erhalten. Beiden wurde die gleiche Subnetzmaske  \glqq 255.255.0.0\grqq \\ übergeben. So kann der Ordner wieder durch die IP-Adresse des Laptops und den Freigabenamen gefunden und in das Dateisystem von dem Raspberry Pi einghängt werden. Mit dieser Variante wird nur ein LAN-Anschluss am Einsatzort des Systems benötigt und eine schnelle und sichere Datenübertragung ermöglicht, sodass wir uns für diese Variante entschieden haben. \\ \\
Durch die Arbeit mit mehreren Raspberry Pis benötigten wir die Einstellung, dass der Ordner für alle Benutzer freigegeben ist. Da der Laptop bei unserer Variante nun mit dem gesamten Netzwerk des Instituts verbunden ist, könnte man als eine zusätzliche Sicherheit (neben dem benötigten Passwort) die Ordnerfreigabe auf einen Benutzer beschränken. Dies müsste noch einmal durch einen Administrator des Laptops geschehen. 

\chapter{Hardwareentwicklung}
\section{Aufbau der Hardware}
TO DO
\section{Steuerung der Motoren}
TO DO
\section{Konstruktion der Halterung}
TO DO

\chapter{Algorithmus zur Werkzeugerkennung}
TO DO

\chapter{Graphische Oberfläche}
TO DO

\chapter{Bedienungsanleitung}
TO DO (erst wenn finale Version fertig ist)


\chapter{Schlusswort}
TO DO (was funktioniert und Aussicht wie System optimiert werden kann)


\end{document}
